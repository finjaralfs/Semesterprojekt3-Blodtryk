\chapter{Baggrund}\label{kapitel_Baggrund}
Udarbejdelsen af gruppens produkt har krævet indsamling og forståelse af en vis viden inden for blodtryk og blodtryksmåling.
Væsentlige dele af denne viden præsenteres her i et afsnit med sundhedsfaglig teori og et med teknologisk teori.

\section{Sundhedsfaglig teori}

Det menneskelige blodtryk er defineret som "(…) den måling, som foretages, når hjertets venstre hjertekammer trækker sig sammen og presser blodet ud i arterierne." (\textit{reference 1}).\\
\newline
Blodtrykket opgives i mmHg, som er trykket i en kviksølvsøjle med den angivne højde. Dvs. hvis blodtrykket er 100 mmHg, betyder det, at trykket er 100 mmHg højere end atmosfæretrykket. 
Det arterielle blodtryk opgives både for det systolisk og det diastoliske tryk. Normalt blodtryk ligger i området 120/80 mmHg, mens 140/90 mmHg angives som grænsen for forhøjet blodtryk og 100/60 mmHg for lavt blodtryk. \\
\newline
Man kan definere blodtrykket vha. formlen: $BT = MV \cdot T$ , altså at blodtrykket er produktet af hjertets minutvolumen (MV) og kredsløbets totale perifermodstand (TPM).
Øget MV, altså den mængde blod hjertet pumper ud gennem arterierne pr. minut, sørger f.eks. for at øge blodtrykket. Jo mere blod der pumpes ud af hjertet pr. minut, jo højere bliver blodtrykket. 
TPM er den modstand blodet møder, når den strømmer gennem blodkarrene. Gennem autonom sympatisk kontrol kan TPM øges, hvorved de glatte muskler omkring arteriolerne kontraheres og blodtrykket stiger (forudsat at MV ikke ændres). Forkalkning i karene resulterer i øget TPM, hvilket bl.a. er grund til, at ældre mennesker tit har forhøjet blodtryk.\\
\newline
Også arteriernes elasticitet har indflydelse på trykket. Dårlig elasticitet i arterierne, altså at væggene ikke giver efter ved øget tryk, vil sørge for at trykket når op på højere værdier, når hjertet pumper blod ud. Arteriernes elasticitet aftager over tid så der opstår stivhed i karrene, som er en anden af grundene til at blodtrykket ofte stiger med alderen.\\
Ud over de ovennævnte forhold, afhænger blodtrykket også af f.eks. psykiske forhold, fordøjelsesaktivitet og fysisk aktivitet. \\

\newpage

\textbf{Hypertension}\\
Hypertension er den kliniske betegnelse for forhøjet blodtryk, og defineres ud fra at blodtrykket har en konstant værdi på omkring 140/90 mmHg og derover. Ved hypertension tales der oftest om arteriel hypertension, dvs. forhøjet blodtryk i arterierne. Den arterielle hypertension kan deles op i to; primær og sekunæd hypertension. Ved den primære hypertension kendes årsagen til det forhøjede blodtryk ikke umiddelbart hvor sekundær skyldes sygdomme i dele af kroppen, der påvirker blodtrykket, eksempelvis hjertet, nyrerne, arterier eller i det endokrine system.\\
Hypertension øger arbejdsbelastningen af hjertet, idet blodet skal pumpes ud af hjertet med en større modstand. Grundet den øgede arbejdsbelastning, øges risikoen for en rækkke sygdomme, heriblandt akut myokardieinfarkt, hjertesvigt og kronisk nyresvigt. \\
Behandlingen af hypertension kan omfatte flere forskellige lægemidler, heriblandt beta-blokkere, ACE-hæmmere, angiotension, II-antagonister og diuretika.


\section{Teknologisk teori}

