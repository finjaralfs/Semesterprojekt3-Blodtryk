\chapter{Baggrund}\label{kapitel_Baggrund}
Udarbejdelsen af gruppens produkt har krævet indsamling og forståelse af en vis viden inden for blodtryk og blodtryksmåling.
Væsentlige dele af denne viden præsenteres her i et afsnit med sundhedsfaglig teori og et med teknologisk teori.

\section{Sundhedsfaglig teori}

Det menneskelige blodtryk er defineret som "(…) den måling, som foretages, når hjertets venstre hjertekammer trækker sig sammen og presser blodet ud i arterierne." (\textit{reference 1}).\\
\newline
Blodtrykket opgives i mmHg, som er trykket i en kviksølvsøjle med den angivne højde. Dvs. hvis blodtrykket er 100 mmHg, betyder det, at trykket er 100 mmHg højere end atmosfæretrykket. 
Det arterielle blodtryk opgives både for det systolisk og det diastoliske tryk. Normalt blodtryk ligger i området 120/80 mmHg, mens 140/90 mmHg angives som grænsen for forhøjet blodtryk og 100/60 mmHg for lavt blodtryk. \\
\newline
Man kan definere blodtrykket vha. formlen: $BT = MV \cdot T$ , altså at blodtrykket er produktet af hjertets minutvolumen (MV) og kredsløbets totale perifermodstand (TPM).
Øget MV, altså den mængde blod hjertet pumper ud gennem arterierne pr. minut, sørger f.eks. for at øge blodtrykket. Jo mere blod der pumpes ud af hjertet pr. minut, jo højere bliver blodtrykket. 
TPM er den modstand blodet møder, når den strømmer gennem blodkarrene. Gennem autonom sympatisk kontrol kan TPM øges, hvorved de glatte muskler omkring arteriolerne kontraheres og blodtrykket stiger (forudsat at MV ikke ændres). Forkalkning i karene resulterer i øget TPM, hvilket bl.a. er grund til, at ældre mennesker tit har forhøjet blodtryk.\\
\newline
Også arteriernes elasticitet har indflydelse på trykket. Dårlig elasticitet i arterierne, altså at væggene ikke giver efter ved øget tryk, vil sørge for at trykket når op på højere værdier, når hjertet pumper blod ud. Arteriernes elasticitet aftager over tid så der opstår stivhed i karrene, som er en anden af grundene til at blodtrykket ofte stiger med alderen.\\
Ud over de ovennævnte forhold, afhænger blodtrykket også af f.eks. psykiske forhold, fordøjelsesaktivitet og fysisk aktivitet. \\

\newpage

\textbf{Hypertension}\\
Hypertension er den kliniske betegnelse for forhøjet blodtryk, og defineres ud fra at blodtrykket har en konstant værdi på omkring 140/90 mmHg og derover. Ved hypertension tales der oftest om arteriel hypertension, dvs. forhøjet blodtryk i arterierne. Den arterielle hypertension kan deles op i to; primær og sekunæd hypertension. Ved den primære hypertension kendes årsagen til det forhøjede blodtryk ikke umiddelbart hvor sekundær skyldes sygdomme i dele af kroppen, der påvirker blodtrykket, eksempelvis hjertet, nyrerne, arterier eller i det endokrine system.\\
Hypertension øger arbejdsbelastningen af hjertet, idet blodet skal pumpes ud af hjertet med en større modstand. Grundet den øgede arbejdsbelastning, øges risikoen for en rækkke sygdomme, heriblandt akut myokardieinfarkt, hjertesvigt og kronisk nyresvigt. \\
Behandlingen af hypertension kan omfatte flere forskellige lægemidler, heriblandt beta-blokkere, ACE-hæmmere, angiotension, II-antagonister og diuretika.


\section{Teknologisk teori}

\section{Baggrund fra klinisk praksis}
\subsubsection{Operationsstue på AUH Skejby}
Det blev bestemt, at projektets produkt skulle være designet til brug på en operationsstue. I den forbindelse blev der taget kontakt til funktionsleder og overlæge Ulf Thyge Larsen på dagkirurgisk afdeling på AUH Skejby. Her blev der arrangeret et en rundvisning af afdelingen med særlig fokus på indretningen i operationsstuen.\\
\newline
Operationsstuen er indrettet med to skærme (1). En skærm til kirurgen, som er nærmest operationsbordet, hvor signaler i form af grafer, talværdier og alarmering for blodtryk, hjerterytme, puls osv. vises. På kirurgens skærm er en fryse funktion, som anvendes til at få et stilbillede af grafen. Derudover er der en alarmeringsfunktion, som bliver aktiveret ved for højt- eller lavt blodtryk. Alarmen vises ved rødt lys på skærmens øverste ramme, samt en alarmerende lyd. Denne lyd kan der skrues ned for, samt den kan gøres lydløs i 3. min ved tryk på knap, så der er mulighed for at kirurgen kan koncentrere sig om grunden til at alarmen blev aktiveret, uden en forstyrrende lyd.\\
\newline
Den anden skærm er til anæstesisygeplejersken, som er koblet til en computer. Her kan der oprettes forbindelse til den elektroniske patientjournal (EPJ) vha. bruger ID og tilhørende password. På denne skærm dokumenterer anæstisisygeplejersken undervejs i operationen.\\
\newline
Der var ved projekts start en ide om en akutfunktion kunne være anvendelig til eventuelle akutte operationer, hvorpå de organisatoriske trin inden en operation kunne springes over. Det blev derfor undersøgt om der var en akutfunktion på operationsstuen - hvor det viste sig, at det var der ikke, idet der altid blev logget ind på patientens EPJ inden en operation. Dette gøres for at have fokus på eventuelle kendte blodtryksværdier, medicin, hjerteproblemer osv. Efter de vigtigste værdier er gennemgået af kirurgen går operationen i gang, og sygeplejersken opdaterer undervejs i operationen med mere info fra EPJ, hvis det er af betydning. \\
\newline
Målingen af blodtrykket vises kontinuerligt ved overvågning under operationen. Selve blodtrykket under operation måles invasivt, hvilket vil sige at der er direkte adgang til patientens kar, hvor patientens arterier så er tilsluttet via et væskefyldt kateter. Posen er under tryk (omkring 250-300 mmHg) og indeholder saltvand. Det er vigtigt at trykket i beholderen er højere end højeste blodtryk, så der kommer et lille flow, der har betydning for, at der ikke staser blod op. Sensoren indeholder en elastisk membran, som kan anvendes til nulpunktsjustering. \\
\newline
Fejlkilder ved blodtryksmålinger kan ske hvis der kommer luftbobler i slangen, hvilket kan være svært at undgå, men væsentligt at minimere. Kalibreringsproblemer kan også være en faktor, der kan give fejlværdier. For at undgå disse fejlværdier, udføres jævnlig kontrol af apparaterne.  
På AUH Skejby var der også kontakt med en medarbejder fra medicoteknisk afdeling, som kunne oplyse om, at kalibrering at blodtryksmålersystemet på operationsstuen sker ca. 1 gang årligt. \\
\newline
\textit{Tidligere er der målt non-invasiv (fokus på høj og lav) vha. stetoskop. …}

