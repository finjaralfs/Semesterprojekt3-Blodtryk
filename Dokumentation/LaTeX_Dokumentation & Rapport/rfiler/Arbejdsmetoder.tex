\chapter{Arbejdsmetoder}\label{kapitel_Arbejdsmetoder}
Under projektarbejdet har gruppen benyttet sig af en række arbejdsmetoder til styring og udvikling.
Her beskrives disse metoder samt en opremsning af de redskaber, gruppen har benyttet.


\section{Udviklingsværktøjer} %*Hed før redskaber
I udviklingen og implementeringen af projektet, er følgende udviklingsværktøjer anvendt:
\subsubsection{DAQ:}
USB-device fra National Instrument. Bruges til at opsamle data som vha. konvertering danner et digitalt signal.
\subsubsection{Analog Discovery:}
Generer et analogt signal ud fra blodtryksdata. Dette analoge signal konverteres, vha. DAQ'en, til et digitalt signal.
\subsubsection{WaveForms:}
Bruges sammen med Analog Discovery og DAQ'en til at indsende data til programmet (???)
\subsubsection{Visual Studio:}
Visual studio er et udviklingsværktøj til software. Dette program er anvendt til at designe software-delen. Programmeringssproget, der er anvendt her, er C$\#$. 
\subsubsection{Microsoft Visio:}
Microsoftprogram til udvikling af hardware-/software diagrammer. Anvendes til at forenkle komplekse oplysninger via enkle, letforståelige diagrammer.
\subsubsection{Multisim:}
Til design af hardware komponeneter til forstærkeren er simuleringsværktøjet Multisim anvendt. 
\subsubsection{Ultiboard:}
Til udlægning af print er programmet Ultiboard benyttet. Det er godt integreret med Multisim, og var derfor det oplagte valg.
\subsubsection{Github:}
Github er et delingsværktøj, som automatisk opretter et versionshistorik. Denne er derfor anvendt til dokumentdeling igennem projektet. 
\subsubsection{Pivotal Tracker:}
Pivotal tracker er anvendt som scrumboard til at styre projektets løbende arbejdsopgaver. 
\subsubsection{Mathad:}
Matchad er et regneværktøj, som er anvendt til at udføre beregner i Hardware-delen. 
\subsubsection{Latex:}
Latex er et såkaldt opmærkningssprog, som er velegnet til rapportskrivning, derfor er dette valgt til projektrapporten. 


\section{Projektstyring}


\section{Udviklingsproces}
