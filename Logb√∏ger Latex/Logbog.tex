\documentclass[a4paper,11pt,oneside]{memoir}

% lidt marginer
	\setlrmarginsandblock{3cm}{*}{1}
	\setulmarginsandblock{3cm}{*}{1}
% mere højde til header
	\setheadfoot{2cm}{\footskip}          
	\checkandfixthelayout[nearest]
% stadard dansk opsaetning
% æøå, andre: ansinew, applemac eller utf8
	\usepackage[utf8]{inputenc} 
% dansk opsætning  
	\usepackage[danish]{babel} 
% fikser babel fejl           
	\renewcommand\danishhyphenmins{22}     
	\usepackage[T1]{fontenc} 
% lidt bedre CM font             
	\usepackage{lmodern}
% god matematik 
	\usepackage{amsmath,amssymb,bm}
	\usepackage{amsthm}
% Billeder
	\usepackage{graphicx}
% farver
	\usepackage{xcolor}
% fjerne automatisk sidetallet hvis kun een side i alt
	\AtEndDocument{%
		\ifnum\value{lastsheet}=1\thispagestyle{empty}\fi}
% lægger info i header i stedet, vi piller ved plain og justerer empty
	\makeoddhead{plain}{Logbøger, gruppe 3}% venstre side
	{Semesterprojekt 3}%
	{\today}% hoejre side 
	%\makeheadrule{plain}{\textwidth}{\normalrulethickness}
	\copypagestyle{empty}{plain}
% vi soerger lige for at fjerne sidetallet ved empty
	%\makeoddhead{empty}{}{}{}
	%\makeoddfoot{empty}{}{}{}
	\pagestyle{plain}
%Matematisk mængde notation
	\newcommand{\N}{\ensuremath{\mathbb{N}}}
	\newcommand{\Z}{\ensuremath{\mathbb{Z}}}
	\newcommand{\Q}{\ensuremath{\mathbb{Q}}}
	\newcommand{\R}{\ensuremath{\mathbb{R}}}
	\newcommand{\C}{\ensuremath{\mathbb{C}}}
	\newcommand{\F}{\ensuremath{\mathbb{F}}}
	\newcommand{\K}{\ensuremath{\mathbb{K}}}
	\newcommand{\mP}{\ensuremath{\mathbb{P}}}
\begin{document}

%-----------------------------------------LOGBOG 1---------------------------------------------------

\begin{center} 
\huge{\textsc{Logbog 1. møde}}
\end{center}

\textbf{ }
\\
\textbf{Logbog:} Møde 1
\\
\textbf{Dato:} 8/9-15
\\
\textbf{Tilstede:} Alle
\\
\textbf{Referent:} Signe
\\
\textbf{Varighed:}
\\

\subsection{Referat:}

\begin{itemize}
\item Vi snakker om at inddele os i grupper - forsøger os med en rolle fordeling
\begin{itemize}
\item To på software
\item To på hardware
\end{itemize}
\item Vi aftaler at Signe er mødereferent - Helle er backup
\item Joakim er ansvarlig for dagsorden
\item \textbf{Aftaler:}
\begin{itemize}
\item Møde en gang om ugen (evt. aftalt dag) - onsdag?
\item Eventuelt fast mødetid med vejleder (spørg Thomas)
\item Samarbejdsaftale lave i morgen, d. 10/9-15, efter møde med Samuel

\end{itemize}
\end{itemize}

\newpage

% --------------------------------------------LOGBOG MØDE 2------------------------------------------

\begin{center} 
\huge{\textsc{Logbog 2. møde}}
\end{center}

\textbf{ }
\\
\textbf{Logbog:} Møde 2
\\
\textbf{Dato:} 9/9-15
\\
\textbf{Tilstede:} Alle
\\
\textbf{Referent:} Signe
\\
\textbf{Varighed:}
\\

\subsection{Referat:}

\begin{itemize}
\item Vi deler os op i en softwaregruppe og en hardwaregruppe
\end{itemize}
\begin{itemize}
\item Tovholder i hver gruppe - og der er så møde en gang i ugen, hvor vi samler trådene. 
\item Hver gang grupperne mødes, skal der skrives en "logbog" over hvad der er lavet, hvilke beslutninger der er taget, og hvor langt mødet er.
\item Der er oprettet Google Docs til deling af dokumenter i opstartsfasen - dokumenterne skal efterhånden skrives ind i Latex.
\item Vi starter med kravspecifikation og use cases
\begin{itemize}
\item Vi går ud fra en blodtryksmåler i en operationsstue
\end{itemize}
\item Vi vil eventuelt gerne kunne gemme i systemet, lave bip-lyde og vise diastolisk og systolisk blodtryk
\item Vi starter med use cases
\item Der skal fremover bruges Github til deling af (Latex-)dokumenter 
\item Trykket omdannes til analog spænding
\item Vi opdeler use cases mellem os
\begin{itemize}
\item Foretag måling: Joakim og Lars
\item Gem data: Rune og Finja
\item Alarmering: Helle og Signe
\end{itemize}
\end{itemize}
\newpage

%-----------------------------LOGBOG 3-------------------------------------------------------------

\begin{center} 
\huge{\textsc{Logbog 3. møde}}
\end{center}

\textbf{ }
\\
\textbf{Logbog:} Møde 3
\\
\textbf{Dato:} 11/9-15
\\
\textbf{Tilstede:} Alle
\\
\textbf{Referent:} Signe
\\
\textbf{Varighed:} 30 minutter
\\

\subsection{Referat:}
\begin{itemize}
\item Vi retter/laver samarbejdsaftale, så denne er helt på plads
\item Vi inddeler os i ansvarsområder:
\begin{itemize}
\item Softwaregruppen: Lars, Joakim og Signe
\item Hardwaregruppen: Rune, Finja og Helle
\end{itemize}
\item Der bliver lavet en dagsorden til mødet mandag d. 14/9-15
\begin{itemize}
\item Tidsplan
\item Arbejdsmetode (Scrum)
\begin{itemize}
\item Hertil lektie: læs op på Scrum
\end{itemize}
\item Use cases
\item Aktørkontekstdiagram
\item Kravspecifikation/accepttest
\end{itemize}
\item Der er aftalt lektier: læs op på Scrum, kig på hardware/software og Latex
\item Samarbejdsaftalen underskrives først i næste uge (mandag/tirsdag)
\end{itemize}

\newpage


%---------------------------------------------LOGBOG 4------------------------------------------------
\begin{center} 
\huge{\textsc{Logbog 4. møde}}
\end{center}

\textbf{ }
\\
\textbf{Logbog:} Møde 4
\\
\textbf{Dato:} 14/9-15
\\
\textbf{Tilstede:} Alle
\\
\textbf{Referent:} Signe
\\
\textbf{Varighed:} 4 timer 
\\

\subsection{Referat:}
\begin{itemize}
\item Vi snakker om arbejdsmetoder og tidsplan - hvor vi vil bruge Scrum 
\begin{itemize}
\item Hvem skal være Scrum-master? Product-owner?
\item Samuel har en platform til oprettelse af Scrum-board.
\end{itemize}
\item Vi laver sprints til kravspecifikation alle sammen, deler op i forhold til hardware og software og sammen til sidst.
\item Vi starter på udkast til systemkrav i dokumentet 'kravspecifikation' i Google Docs
\item Vi gennemgår sammen systemkrav
\item Skal vi have én eller to skærme?
\item Spørgsmål til vejleder:
\begin{itemize}
\item Digital filtrering?
\end{itemize}
\item Vi påbegynder at opskrive ikke-funktionelle krav
\item Vi har udviklet et forslag til use cases, påbegyndt aktørkontekstdiagrammer- og beskrivelser samt use case-diagrammer
\item Vi har fået adgang til Pivotal Tracker
\item Vi deler op hvad der skal gøres og hvem? 
\item Vi skal have lavet aktørbeskrivelser 
\item Samuel giver introduktion til Pivotal Tracker
\item Vi laver en tidsplan (samlet)
\item Vi aftaler en dato til møde med vejleder (Thomas) og laver en dagsorden dertil 
\begin{itemize}
\item Introduktion til projektet/vejleders holdning
\item Kravspecifiktaion
\item Use cases
\item Tidsplan
\end{itemize}
\end{itemize}

\newpage

%------------------------------------------LOGBOG 5------------------------------------------------
\begin{center} 
\huge{\textsc{Logbog 5. møde}}
\end{center}

\textbf{ }
\\
\textbf{Logbog:} Møde 5
\\
\textbf{Dato:} 21/9-15
\\
\textbf{Tilstede:} Joakim, Lars, Rune, Helle og Signe 
\\
\textbf{Referent:} Signe
\\
\textbf{Varighed:} 1 time
\\

\subsection{Referat:}
\begin{itemize}
\item Det diskuteres hvorledes 'Hent data' skal have sin egen UC eller være en del af en allerede eksisterende 
\item Login skal bestå af et simpelt brugernavn samt kode 
\item Lav eventuelt en 'gæstelogin' til hurtig måling 
\item UC's:
\begin{itemize}
\item UC1: Login 
\item UC2: Start måling
\item UC3: Gem måling
\end{itemize}
\item Der skal laves noget, så et allerede kendt blodtryk kan reguleres 
\end{itemize}
\newpage

%---------------------LOGBOG 6-----------------------------

\begin{center} 
\huge{\textsc{Logbog 6. møde}}
\end{center}

\textbf{ }
\\
\textbf{Logbog:} Møde 6
\\
\textbf{Dato:} 23/9-15
\\
\textbf{Tilstede:} Alle
\\
\textbf{Referent:} Signe
\\
\textbf{Varighed:} 2 timer
\\

\subsection{Referat:}
\begin{itemize}
\item Helle har snakket med onkel og søster, vi tager eventuelt derud på Skejby og ser på en OP-stue
\item Vi snakker kort om opstart på programmering - største problem bliver eventuelt at få DAQ'en tilkoblet programmet 
\item Systembeskrivelse skal tegnes ind, samt billede af systemet 
\item KS rettes til 
\item Der udvikles skitser til de forskellige trin - som lægges ind i Latex
\item Vi tilretter UC's
\item Mødes igen mandag d. 28/9 fra 10-12 og får rettet det sidste til
\item Vi tager ud og besøger Skejby i morgen, torsdag efter kl. 14 
\item Møde med vejleder tirsdag d. 29/9 kl. 12.15
\end{itemize}

%\begin{itemize}
%\item Visere, impedans/admittans
%\item Knudepunkts- og maskeligninger
%\begin{itemize}
%\item Første underpunkt
%\item Andet underpunkt
%\end{itemize}
%\end{itemize}
\newpage

%---------------------LOGBOG 7-----------------------------

\begin{center} 
\huge{\textsc{Logbog 7. møde}}
\end{center}

\textbf{ }
\\
\textbf{Logbog:} Møde 7
\\
\textbf{Dato:} 28/9-15
\\
\textbf{Tilstede:} Alle
\\
\textbf{Referent:} Signe
\\
\textbf{Varighed:} 2 timer
\\

\subsection{Referat:}
\begin{itemize}
\item Helle starter med at fortælle om besøg på Skejby (se PDF)
\begin{itemize}
\item De har to skærme på OP, én til kirurgen og én til anestæsisygeplejersken
\item Akutfunktion ikke nødvendig, de \textit{skal} nå at logge ind, inden en operation
\item Lægen kan trykke på en "pause"-knap, hvor alarmen stopper i 3 minutter, hvor lægen så har de 3 minutter til at redde situationen
\item Lyd er vigtigt, for at lægen opdager det
\item Fryse-funktion
\item Blodtrykket går ud fra baseline (ca. 1/3 over det laveste blodtryk)
\item Der måles altid invasivt og opad)
\item Trykket i væskebeholderen skal være lidt over det øverste blodtryk, for at der ikke stadser blod op i måleapperatet 
\item Automatisk kalibrering 
\item Fejlkilde: Luftbobler i slangen
\end{itemize}
\item Vi laver eventuelt en UC til kalibrering 
\item Akut-funktionen slettes 
\item Der skal laves en alarmlyd (evt. noget lys også)
\item Overvejelserne før og efter snak med Ulf Thyge Larsen (overlæge på dagkirurgisk) og sammenligne med resultater 
\item Rune har skrevet noget kode, som han gennemgår - vi skal være enige omkring, hvad vi vil have 
\begin{itemize}
\item Idegenerering af Chart i programmet
\end{itemize}
\item Vi retter kravspecifikation til (har oprettet et forprojekt med de gamle ideer)
\item Finja læser om Github - som vi forhåbentlig kan bruge fra i morgen
\item Møde med vejleder i morgen d. 29/9-15
\end{itemize}

\newpage

%---------------------LOGBOG 8-----------------------------

\begin{center} 
\huge{\textsc{Logbog 8. møde}}
\end{center}

\textbf{ }
\\
\textbf{Logbog:} Møde 8
\\
\textbf{Dato:} 29/9-15
\\
\textbf{Tilstede:} Alle
\\
\textbf{Referent:} Signe
\\
\textbf{Varighed:} 
\\

\subsection{Referat:}
\begin{itemize}
\item Rettelser efter møde med vejleder
\item Tekniker som sekundær i UC-diagram (han er sekundærrolle i forhold til systemets mål) og primær i UC 
\end{itemize}

\newpage

%---------------------LOGBOG 9-----------------------------

\begin{center} 
\huge{\textsc{Logbog 9. møde}}
\end{center}

\textbf{ }
\\
\textbf{Logbog:} Møde 9
\\
\textbf{Dato:} 30/9-15
\\
\textbf{Tilstede:} Lars, Joakim og Signe
\\
\textbf{Referent:} Signe
\\
\textbf{Varighed:} 
\\

\subsection{Referat:}
\begin{itemize}
\item Rettelse af accepttest
\item Skal password indgå som **** ?
\item Funktionelle samt ikke-funktionelle krav er stillet op
\end{itemize}

\newpage


%---------------------LOGBOG 10-----------------------------

\begin{center} 
\huge{\textsc{Logbog 10. møde}}
\end{center}

\textbf{ }
\\
\textbf{Logbog:} Møde 10
\\
\textbf{Dato:} 5/10-15
\\
\textbf{Tilstede:} Alle
\\
\textbf{Referent:} Signe
\\
\textbf{Varighed:} 
\\

\subsection{Referat:}
\textbf{Review af gr. 4}
\begin{itemize}
\item \textbf{Ikke-funktionelle krav (og generelt)}
\begin{itemize}
\item Kald eventuelt Thomas til projektvejleder - og skriv gerne efternavnet på
\item Husk og rette ansvarsområder - det er ikke jeres navne - og hvilke initialer har hvem?
\item Genovervej indledning - uddyb mere eller forkort og gør mere præcis?
\item Ikke-funktionelle krav skal eventuelt komme efter funktionelle krav - man kastes direkte ind i systemet, uden at have nogen idé om aktører osv.
\item Små rettelser til ikke-funktionelle krav - se noter
\item Angiv eventuelt krav med tal, så disse kan henvises til
\end{itemize}
\item \textbf{Funktionelle krav}
\begin{itemize}
\item Aktør-kontekst diagram under overskriften, og der mangler figurtekst
\item Bruger/sundhedsfagligt personale?
\item Use case diagram - figurtekst
\item Skal data ikke hentes? I så fald skal database have kontakt med opstart system
\item UC1: Udfyld kalibrering
\item UC2: Hvad menes med samtidige forløb? Log-in eller login? Knapper i UC skal angives i ikke-funktionelle krav? Hvilke login-oplysninger skal bruges - og skal der så være adgang til databasen? Skriv eventuelt hvad målet er for use casen? Undtagelse 2.a, hvordan (nulpunktsjusterings-knap overflødig)???
\item UC3: Initialisere, er det ikke målet? Resultat/mål (resultat starter med 'at'). Beskrivelse af alarmering? 'Knappen' er stavet forkert hele vejen igennem.
\item UC4: Skal UC3 være færdig før UC4 starter?
\item UC5: Samtidige forekomster? Bruger justerer grænseværdier - hvordan? Resultat/mål?
\item UC6: Resultat er godt! 60 sekunder er ikke længe? Hvad sker der efter trykket på 'udskyd alarm'?
\item UC7: Initialisere/mål? Undtagelse 2.a skal eventuelt komme efter pkt. 3 - hvordan ved computeren lige efter indtastning, at CPR er forkert? Hvad menes med UC3 i undtagelser?
\end{itemize}
\item \textbf{Accepttest use cases}
\begin{itemize}
\item Lav eventuelt test af ikke-funktionelle krav først, da det er det der kommer først i kravspecifikation
\item UC2: Krav stemmer ikke overens med accepttest - brugeroplysninger? Log ind på en tredje måde. OK-vindue indgår ikke i UC2 i kravspecifiktaion? Test af UC2 og undtagelser sammen? 
\item UC3: Undtagelse 1.a, hvad er de værdier? Pil op/ned???
\item UC4: God, men hvor påsættes sinussignal?
\item UC5: Er resultatet 'bare' at det står i tekstfeltet? Sker der ikke noget i grafen?
\item UC6: Fin
\item UC7:  
\end{itemize}
\item \textbf{Accepttest af ikke-funktionelle krav}
\begin{itemize}
\item Krav 4: hvordan?
\item 'At' i forventet resultat?
\item Mangler resultat
\item Reliability: det kan i ikke teste?
\item Der skal stå hvad der skal gøres, og ikke bare "login"
\end{itemize}
\item Generelt: kravspecifikation og accepttest skal stemme overens, ellers rigtig fint!
\item Ret initialisere i UC1 (vores egen. 
\end{itemize}

\newpage

%---------------------LOGBOG 11-----------------------------

\begin{center} 
\huge{\textsc{Logbog 11. møde}}
\end{center}

\textbf{ }
\\
\textbf{Logbog:} Møde 11
\\
\textbf{Dato:} 20/10-15
\\
\textbf{Tilstede:} Helle, Lars, Finja, Joakim og Signe
\\
\textbf{Referent:} Signe
\\
\textbf{Varighed:} 
\\

\subsection{Referat:}
\begin{itemize}
\item HW-gruppen har i ferien lavet andenordensfiltret
\item Mail fra Thomas med smårettelser fra ferien gennemgås, og de nødvendige rettelser foretages
\item Latex dokumenterne (Logbog, dokumentation, mødereferat) ligges ind i Github 
\item Det besluttes, at lave en ny UC3 - filtrer signal. Denne oprettes og laves
\item Brian gennemgår programmeringen
\begin{itemize}
\item Asynkroncallback - vigtig metode til, at få signalet kørt ind asynkront
\end{itemize}
\end{itemize}

\newpage

%---------------------LOGBOG 12-----------------------------

\begin{center} 
\huge{\textsc{Logbog 12. møde}}
\end{center}

\textbf{ }
\\
\textbf{Logbog:} Møde 12
\\
\textbf{Dato:} 22/10-15
\\
\textbf{Tilstede:} Alle
\\
\textbf{Referent:} Signe
\\
\textbf{Varighed:} 
\\

\subsection{Referat:}
\begin{itemize}
\item UC3 gennemgås - der skal laves accepttest og UC-diagram
\item HW-gruppen gennemgår det de har lavet i ferien 
\begin{itemize}
\item Forløbet rimelig uden problemer, der er spørgsmål til Thomas (operationsforstærkers selvinduktion?)
\item Lavpasfilteret er første udkast lavet, forstærkeren mangler 
\end{itemize}
\item Software-gruppen
\begin{itemize}
\item Gennemgår diagrammerne der blev lavet i går, domæne, applikationsmodel
\item Rune kommer evt. mere ind over SW 
\end{itemize}
\item BDD og IBD skal udvikles snart, evt. i samarbejde efter møde med Thomas - lavet evt. som udkast til mødet i morgen
\item Hardware-gruppe laver IBD, og BDD
\item Software-gruppe laver sekvensdiagram
\end{itemize}

\newpage

%---------------------LOGBOG 13-----------------------------

\begin{center} 
\huge{\textsc{Logbog 13. møde}}
\end{center}

\textbf{ }
\\
\textbf{Logbog:} Møde 13
\\
\textbf{Dato:} 29/10-15
\\
\textbf{Tilstede:} Lars, Joakim og Signe (SW)
\\
\textbf{Referent:} Signe
\\
\textbf{Varighed:} 2 timer
\\

\subsection{Referat:}
\begin{itemize}
\item Tilretter domænemodel - omdøber alle data, fra SP til GUI, til \textit{GUI data}, og samme fra GUI til System
\item Sekvensdiagram over hele systemet påbegyndes. Dette skrives uden specifikke metoder, hvor vi efter start på programmering laver et sekvensdiagram pr. UC og skriver metoder til 
\end{itemize}

\newpage

%---------------------LOGBOG 14-----------------------------

\begin{center} 
\huge{\textsc{Logbog 14. møde}}
\end{center}

\textbf{ }
\\
\textbf{Logbog:} Møde 14
\\
\textbf{Dato:} 4/11-15
\\
\textbf{Tilstede:} Alle
\\
\textbf{Referent:} Signe
\\
\textbf{Varighed:} 4 timer
\\

\subsection{Referat:}
\begin{itemize}
\item Holder møde med Torben:
\begin{itemize}
\item Transducer skal være sekundær, patient skal ikke være i systemet
\item Arkitektur til aflevering (HW): BDD, IBD, signalbeskrivelse 
\item Domænemodel, applikationsmodel og sekvensdiagram for hver enkelt use case
\end{itemize}
\item Kravspecifikation tilrettes fra bunden 
\item Use case to laves om i forhold til tranducer og nulpunktsjustering
\item Kravspecifiktion og accepttest er rettet til efter kontraordre
\end{itemize}

\newpage

%\begin{figure}[h]
%\centering
%\includegraphics[scale=0.45]{eventhandler.PNG}
%\caption{Eventhandlere}
%\end{figure}

%$$\sin^2 s + \cos^2 s = 1 \ \forall s \in \R$$

\end{document}